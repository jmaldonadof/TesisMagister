%
%Introducción
%	Contexto y motivación
%	Marco Teórico (Resumido)
%	Objetivos
%	Metodología de trabajo 
%	Principales desafíos
%	Resultados
%	Organización del documento

\chapter{Introducción}

\section{Contexto}
Para muchos países, es muy importante estar preparados en caso de que ocurra un sismo y de esta forma evitar el daño producido.
%
Países como Japón, China, Estados Unidos y Chile, entre otros, invierten gran cantidad de recursos en el estudio de terremotos y hay una constante necesidad de métodos novedosos para recolectar información rápidamente durante los momentos de crisis.
%
Actualmente la principal fuente de información consiste en redes de sensores sísmicos de alta sensibilidad, que son capaces de detectar movimientos imperceptibles por las personas. 
%
Mediante el uso de sensores se puede medir la magnitud de un sismo, la que se relaciona con la cantidad de energía liberada en forma de ondas. 
%
Con la información proveniente de estos sensores se puede determinar, en cierto grado, la forma en que un evento sísmico es percibido en diferentes regiones de la superficie terrestre, lo que se conoce como intensidad del sismo. 
%
Sin embargo, esta estimación no es muy exacta, ya que la intensidad no sólo depende de la fuerza del sismo (magnitud) sino que también de la distancia epicentral, la geología local, la naturaleza del terreno y el tipo de construcciones del lugar.
%
Para obtener información más certera se requiere que personal capacitado valide que un sismo percibido por un sensor corresponde a un sismo sensible\footnote{Sismo percibido por personas en alguna parte} o no. 
%
En Chile esta tarea recae en voluntarios de las oficinas de emergencia nacionales (ONEMI)\footnote{http://www.onemi.cl/}, quienes envían un reporte con la intensidad que percibieron en el área geográfica en la que se encuentran; 
%
y en otros países existen iniciativas que, por medio del \textit{crowdsourcing}\footnote{La práctica de obtener información o aportes en una tarea mediante la colaboración de personas, ya sea mediante un contrato remunerado o de manera voluntaria y generalmente a través de internet.}, determinan si el sismo fue perceptible~\cite{atkinson2007did}.
%
Esta información es valiosa porque permite a sismólogos estimar de mejor manera la fuerza con que un sismo fue percibido, el daño causado y el área de impacto, así como también establecer relaciones entre el tamaño de un sismo y los factores antes mencionados. 


Cabe destacar que, sin importar si un sismo es de alta o baja magnitud, su estudio y caracterización, le permite a los expertos obtener conocimiento sobre la actividad sísmica de ciertas regiones y construir catálogos muy completos sobre sismos en el mundo~\cite{stein2009introduction}.
%
Las oficinas de emergencia nacionales se esfuerzan en determinar la intensidad de un sismo en todas los lugares donde este fue percibido, incluso si se trata de un sismo pequeño. 
%
Este conocimiento permite mejorar las políticas de respuesta ante desastres y la estimación del impacto que podrían causar sismos de mayor magnitud en la misma área.
%
Finalmente y a modo de observación, los sensores sísmicos son artefactos costosos que además implican un alto costo de instalación y mantención, por lo que existen zonas en el mundo que no tienen buena cobertura. Esto puede dificultar la creación de catálogos de sismos suficientemente completos~\cite{usgs,shoa,onemi,wyss2009delay}.


En esta tesis se propone la utilización de una fuente de información no convencional, para desarrollar una herramienta de soporte de decisiones durante eventos sísmicos.
%
En particular, el trabajo se basa en la utilización de Twitter\footnote{\url{https://www.twitter.com}}, una red social de microblogging, la cual se caracteriza por sus mensajes de máximo 140 caracteres (llamados {\em tweets})\footnote{El 8 de Noviembre del 2017 el número de caracteres ha aumentado a 280.}.
%
Esta red social es utilizada actualmente por más de 328 millones de usuarios activos mensuales~\cite{abouttwitter} que publican mensajes y propagan información en tiempo real, abarcando desde noticias de última hora sobre eventos muy importantes hasta actualizaciones mundanas y personales.
%
Además, alrededor del 80\% de los usuarios de Twitter acceden al servicio utilizando sus dispositivos móviles, lo que contribuye a la inmediatez de la difusión de la información. 
%
Por estas razones, Twitter se ha convertido en la fuente de información preferida por periodistas y usuarios en general, sobretodo durante momentos de crisis y desastres naturales~\cite{castillo2016bcd,Mendoza:2010:TUC:1964858.1964869}.


\section{Estado del Arte}

En su influyente trabajo, Sakaki et al.~\cite{sakaki2010earthquake} mostraron que los usuarios de redes sociales (en particular usuarios de \textit{Twitter}) pueden ser usados para detectar rápidamente la ocurrencia de sismos e incluso para estimar el epicentro.
%
Los usuarios de redes sociales, o los llamados ``sensores ciudadanos'' o ``sensores sociales''~\cite{sheth2009citizen} entregan información por muy bajo costo y generan una cantidad enorme de datos sobre lo que viven en su día a día. 
%
Desde ese momento, varios estudios han abordado el problema de detección y descripción de sismos usando datos de redes sociales~\cite{avvenuti2014ears,cameron2012emergency,earle2010omg,earle2012twitter}. 
%
En particular, Young et al.~\cite{young2013transforming} presentaron un caso de estudio de la necesidad de aprovechar los datos de sensores sociales en la investigación sismológica, y cómo esto puede puede proveer información de calidad de bajo costo para áreas que no están bien cubiertas por los métodos existentes de recolección de información.


A pesar de la utilidad de usar los datos de redes sociales para la detección y descripción de sismos, la revisión al estado del arte muestra que el problema no está resuelto en términos de precisión, {\em recall} y cobertura geográfica~\cite{young2013transforming}.
%
Las propuestas existentes tienen alta precisión para eventos de gran magnitud, los cuales son percibidos por muchas personas, pero bajo {\em recall} cuando se considera el rango completo de sismos sensibles.
%
Esto ocurre en algunos casos debido al ruido presente en los datos (i.e., mensajes irrelevantes), lo que implica que los sistemas sólo alertan en casos de alta confiabilidad, evitando con esto, falsos positivos~\cite{earle2010omg,earle2012twitter,avvenuti2014ears}.
%
Ciertos sistemas aumentan la cantidad de sismos sensibles detectados, pero a expensas de limitar la cobertura geográfica a nivel de país, usando enfoques supervisados y filtros estrictos sobre los datos de entrada~\cite{avvenuti2014ears,sakaki2013tweet}.
%
Esos enfoques son extremadamente difíciles de escalar o replicar a otros países, debido al alto costo de adaptación de los modelos y el etiquetamiento de los datos requeridos.
%
Como consecuencia, no hay sistemas que logren un buen equilibrio entre la precisión y el {\em recall} para la detección de sismos en múltiples regiones e idiomas. 
%
Tampoco hay sistemas disponibles públicamente para detectar sismos en el mundo en tiempo real (y describirlos) usando sensores sociales. 


\section{Objetivos}
\label{sec:objetivos}

\subsection{Objetivo General}
	
\begin{itemize}

\item Implementación de un sistema que, utilizando datos obtenidos desde \textit{Twitter}, detecte la ocurrencia de sismos en tiempo real y visualice información útil extraída de los datos.\\ 

\end{itemize}

\subsection{Objetivos Específicos}

\begin{enumerate}

\item Proveer una metodología no supervisada que permita detectar de forma eficiente y efectiva la ocurrencia de sismos en Chile y en el mundo usando información publicada en \textit{Twitter}.

\item Comparar la metodología propuesta con el estado del arte en detección de sismos usando Twitter. 

\item Implementar un sistema que utilice la metodología propuesta para detectar sismos automáticamente.

\item Extraer información relevante a partir de los datos que permitan caracterizar un sismo, por ejemplo, datos geográficos.

\item Almacenar la información para estudios posteriores y para explorar datos de eventos pasados.

\item Implementar una aplicación para visualizar la información en tiempo real que sirva de apoyo para el Centro Sismológico Nacional\footnote{http://www.sismologia.cl} y otras organizaciones o personas que deseen ocuparla.

\end{enumerate}


\section{Metodología de Trabajo}

Para la realización de este trabajo, se utilizó una metodología ágil que permitió implementar prototipos funcionales, los cuales eran testeados por los usuarios expertos (Sismólogos del Centro Sismológico Nacional), mientras tanto, en paralelo se realizaban los análisis experimentales que validan la tesis. Además esto permitió incorporar mejoras basadas en las opiniones de los usuarios para luego difundir la aplicación para un uso masivo. 

El trabajo se divide en tres etapas importantes. La primera etapa está enfocada en el desarrollo de una primera versión del sistema de detección de sismos y de la aplicación Web que visualiza la información obtenida en tiempo real. El fruto de este trabajo es un prototipo básico que permite comenzar con la recolección de datos y brinda a los usuarios un primer producto valioso. La segunda etapa se enfoca en el estudio y mejora del sistema de detección automática y en el estudio de los datos recolectados. Esta segunda etapa permite validar la eficacia de la detección y mejorar la eficiencia. La tercera etapa consiste en extender la aplicación Web para incorporar la opción de explorar eventos pasados e incorporar mejoras en base a los resultados del análisis experimental y de las opiniones entregadas por los usuarios durante el periodo de pruebas.

A continuación se detallan las tareas principales abordadas en cada etapa.

%\subsubsection*{Primera Etapa}
\textbf{Primera Etapa}
uguz
\begin{itemize}

\item Puesta en marcha de un sistema de recolección continua de \textit{tweets} que contengan la palabra sismo, temblor o terremoto en diferentes idiomas. 
\item Adaptación del detector de ráfagas a partir del \textit{stream} de \textit{Twitter} que proponen J. Guzmán y B. Poblete \cite{guzman2013line} para su utilización sobre los datos recolectados sobre sismos.
\item Implementación de una aplicación Web que visualice en tiempo real los datos recolectados a través de una serie temporal de frecuencia de \textit{tweets}, un listado con los \textit{tweets} recolectados y un mapa con los \textit{tweets} y usuarios que tengan información geográfica.
\item Inicio del periodo de marcha blanca en la que los usuarios expertos validan el prototipo.

\end{itemize}

%\subsubsection*{Segunda Etapa}
\textbf{Segunda Etapa}

\begin{itemize}
%\setcounter{enumi}{4}

\item Análisis de los datos recolectados para mejorar el desempeño del sistema de detección de sismos. 
\item Análisis experimental utilizando los datos recolectados y comparando las detecciones automáticas generadas con reportes oficiales de oficinas de sismología.
\item Mejoras en la geocodificación de los datos para mejorar las visualizaciones geográficas. 

\end{itemize}

\textbf{Tercera Etapa}

\begin{itemize}
%\setcounter{enumi}{6}
\item Incluir información de detecciones automáticas a la aplicación Web.
\item Incorporar mejoras basadas en las opiniones de los usuarios y comportamiento observado durante la marcha blanca.
\item Incorporar la opción de explorar eventos pasados a la aplicación Web.

\end{itemize}

\section{Principales Desafíos}

El principal desafío consistió en lograr que la solución fuera tolerante al ruido de los datos y de esta forma permitiera detectar sismos en diferentes lugares del mundo. 
%
Lograr un equilibrio entre la precisión y el {\em recall} para sismos de diferente magnitud y en diferentes lugares del mundo es una tarea que hasta el momento no había sido resuelta utilizando las técnicas propuestas del estado del arte. 
%
Soluciones anteriores necesitaban limitar el conjunto de datos de entrada a un idioma específico o una región específica, y con ello lograban obtener buenos resultados para la zona específica que se deseaba abarcar. 
%
En cambio, en esta tesis se apostó por un enfoque genérico, que fuera útil para el caso chileno, pero que también entregase información sobre sismos ocurridos en el resto del mundo.


Otro desafío fue la disminución los costos de parametrización y de puesta en marcha, es decir, que no fuera necesario contar con un conjunto de datos etiquetados o un estudio profundo de los datos de entrada para su puesta en marcha.
%
Las soluciones del estado del arte con mejores resultados utilizan enfoques supervisados que requieren tener conjuntos de datos etiquetados. 


Un tercer desafío fue lograr que el tiempo de detección se mantuviera bajo. Considerando que el objetivo es brindar soporte durante momentos de crisis, es muy importante que la información recolectada sea procesada y puesta a disposición de los usuarios finales rápidamente.


\section{Resultados Obtenidos}

El método de detección de sismos propuesto es tolerante al ruido, no supervisado y fácil de parametrizar. 
%
Es robusto en términos de proveer buena precisión y {\em recall} para sismos de alta y baja magnitud que son percibidos por las personas.
%
El enfoque es simple, requiere pocos recursos computacionales y permite realizar la detección para múltiples idiomas a la vez.


Se implementó un sistema de visualización Web que utiliza la información proveniente de sensores sociales y marca los eventos ocurridos en base a las detecciones realizadas por el algoritmo de detección. 
%
La herramienta bautizada como Twicalli\footnote{término acuñado por M. Strohmaier~\cite{Strohmaier2010Twicalli}.\\Sistema disponible en \url{http://twicalli.cl}}, 
se encuentra públicamente disponible y es utilizada actualmente, como herramienta de soporte de decisiones, por el Centro Sismológico Nacional (CSN) y por el Servicio Hidrográfico y Oceanográfico de la Armada Chilena (SHOA).


El método propuesto basa su análisis en el procesamiento de señales creadas mediante la agregación de los mensajes relacionados con sismos. 
%
El procesamiento considera, entre otros, la velocidad de llegada de los mensajes asociados a diferentes localidades mencionadas en el texto del mensaje o localidades asociadas al usuario.
%
Esto permite al usuario del sistema, por ejemplo, desambiguar entre diferentes eventos que ocurren simultáneamente en diferentes lugares del mundo o separados por un lapso mínimo de tiempo.
%
Esta desambiguación no es fácil de realizar usando sismógrafos~\cite{kennett1991traveltimes,sambridge2001seismic}.


Se realizó una evaluación cuantitativa del algoritmo de detección sobre un periodo de 9 meses usando varios criterios y catálogos de datos reales de sismos. 
%
La evaluación realizada es la más larga realizada hasta la fecha para este tipo de sistemas.


Los resultados obtenidos son muy competitivos, alcanzando una precisión de $0.99$ y \textit{recall} de $0.85$ (\textit{F-measure} de $0.91$) para sismos con magnitud $\geq 4.0$ que fueron percibidos por las personas. 
%
Para sismos de magnitud $\leq 4.0$, la mayoría de ellos no sentidos por las personas (y por lo tanto no detectados por sensores sociales), se alcanzó una precisión de $1.00$ y \textit{recall} de $0.15$ (F-measure de $0.26$).
%
Estos resultados superan significativamente a la mayoría de los resultados reportados por otros sistemas supervisados y no supervisados de estudios similares.
%
En el único caso en que el sistema propuesto obtiene resultados cercanos, pero que no superan al sistema con el cual se compara, se compensa debido a que el sistema propuesto detecta sismos en todo el mundo de manera no supervisada mientras que el sistema con el cual se compara lo hace de forma supervisada y en un solo país.


\section{Organización del Documento}

En el capítulo \ref{cap:marco} se definen algunos conceptos utilizados a lo largo de este documento y se presenta el marco teórico de forma detallada. 
%
Luego, en el capítulo \ref{cap:deteccion} se explica el algoritmo de detección de sismos implementado. 
%
En el capítulo \ref{cap:arquitectura} se describe la arquitectura completa del sistema desarrolado. 
%
En los capítulos \ref{cap:sentimiento} y \ref{cap:geocodificacion} se describen brevemente el trabajo realizado en relación con análisis de sentimiento de los \textit{tweets} y geolocalización. 
% 
En el capítulo \ref{cap:aplicacion} se presenta la aplicación Web desarrollada y las visualizaciones utilizadas para presentar la información. 
%
En el capítulo \ref{cap:analisis} se describen la preparación, desarrollo y resultados de los experimentos realizados para validar el sistema de detección. 
%
En el capítulo \ref{cap:casos} se presentan algunos casos de sismos detectados y su exploración mediante la aplicación.
%
Finalmente en el capítulo \ref{cap:conclusion} se discuten los resultados obtenidos y posible trabajo futuro.  